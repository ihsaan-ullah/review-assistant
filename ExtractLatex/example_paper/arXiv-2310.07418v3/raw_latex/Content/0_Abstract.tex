Plasticity, the ability of a neural network to evolve with new data, is crucial for high-performance and sample-efficient visual reinforcement learning (VRL). Although methods like resetting and regularization can potentially mitigate plasticity loss, the influences of various components within the VRL framework on the agent's plasticity are still poorly understood. In this work, we conduct a systematic empirical exploration focusing on three primary underexplored facets and derive the following insightful conclusions: (1)~data augmentation is essential in maintaining plasticity; (2)~the critic's plasticity loss serves as the principal bottleneck impeding efficient training; and (3)~without timely intervention to recover critic's plasticity in the early stages, its loss becomes catastrophic.
These insights suggest a novel strategy to address the high replay ratio (RR) dilemma, where exacerbated plasticity loss hinders the potential improvements of sample efficiency brought by increased reuse frequency.
Rather than setting a static RR for the entire training process, we propose \textit{Adaptive RR}, which dynamically adjusts the RR based on the critic’s plasticity level.
Extensive evaluations indicate that \textit{Adaptive RR} not only avoids catastrophic plasticity loss in the early stages but also benefits from more frequent reuse in later phases, resulting in superior sample efficiency.

