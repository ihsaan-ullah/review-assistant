\usepackage{multirow}
\usepackage{amsmath}
\usepackage{capt-of}
\usepackage{tabularx}
% \usepackage[caption=false]{subfig}
\usepackage{epsfig}
\usepackage{amssymb}
\usepackage{amsfonts}
\usepackage{booktabs}
\usepackage{scalerel}
%\usepackage[dvipsnames]{xcolor}
\usepackage[inline]{enumitem}
\usepackage{listings}
\usepackage{varwidth}
\usepackage[export]{adjustbox}
\usepackage{tikz}
\usetikzlibrary{tikzmark}
% \usepackage{todonotes}
% \usepackage{cleveref}
\newcommand{\crefrangeconjunction}{--}
\usepackage{stmaryrd}
\usepackage{bbm}
\usepackage{wrapfig}
\usepackage{pifont}

\newcommand{\tabincell}[2]{\begin{tabular}{@{}#1@{}}#2\end{tabular}}
\newcommand{\tx}[1]{``\textit{#1}''}
\newcommand{\sptk}[1]{\texttt{[#1]}}
% \newcommand{\eqform}[1]{Equation~(\ref{#1})}

% \usepackage{algorithm}
% \usepackage[noend]{algpseudocode}

\definecolor{deepblue}{rgb}{0,0,0.5}
\definecolor{officeblue}{RGB}{0,102,204}
\definecolor{deepred}{rgb}{0.6,0,0}
\definecolor{deepgreen}{rgb}{0,0.5,0}
\definecolor{mybrickred}{RGB}{182,50,28}
\newcommand\mybox[2][]{\tikz[overlay]\node[inner sep=1pt, anchor=text, rectangle, rounded corners=0mm,#1] {#2};\phantom{#2}}
\definecolor{fillcolor}{RGB}{216,217,252}
\newcommand\bg[1]{\mybox[fill=blue!20]{#1}}
\newcommand\rg[1]{\mybox[fill=red!20]{#1}}
\newcommand\graybox[1]{\mybox[fill=gray!20]{#1}}

% %%%%%Algorithm Box Package%%%%%
% \renewcommand{\algorithmicrequire}{\textbf{Input:}}
% \algnewcommand\algorithmicrequireb{{\hspace{0.85cm}}}
% \algnewcommand\INPTDESCB{\item[\algorithmicrequireb]}
% \renewcommand{\algorithmicensure}{\textbf{Output:}}
% \algnewcommand\algorithmicfuncdesc{\textbf{Function:}}
% \algnewcommand\FUNCDESC{\item[\algorithmicfuncdesc]}
% \algnewcommand\algorithmicfuncdescb{{\hspace{1.48cm}}}
% \algnewcommand\FUNCDESCB{\item[\algorithmicfuncdescb]}
% \algnewcommand{\algorithmicgoto}{\textbf{goto}}
% \algnewcommand{\Goto}[1]{\algorithmicgoto~\ref{#1}}
% \newcommand*\Let[2]{\State {#1 $\gets$ #2}}
% \newcommand*\LineLet[2]{#1 $\gets$ #2}
% %\newcommand*\AddOne[1]{\State #1 $\gets$ #1 $+1$}
% \newcommand*\AddOne[1]{\State #1 $++$}
% \newcommand*\LineComment[1]{\Statex \(\triangleright\) #1}
% \newcommand*\LineFor[2]{\State {\algorithmicfor~#1~\algorithmicdo~~~~#2}}
% \newcommand*\LineIf[2]{\State {\algorithmicif~#1~\algorithmicthen~~~~#2}}
% \newcommand*\AlgCommentInLine[1]{{\color{deepblue}{$\triangleright$ \textit{#1}}}}
% \newcommand*\AlgComment[1]{\State{\AlgCommentInLine{#1}}}
% %%%%%Algorithm Box Package%%%%%

\newcommand*\AlgCommentInLine[1]{{\color{deepblue}{$\triangleright$ \textit{#1}}}}


% \usepackage{ifxetex,ifluatex}
\usepackage{etoolbox}
% \usepackage[svgnames]{xcolor}
% \usepackage{tikz}
\usepackage{framed}

% conditional for xetex or luatex
\newif\ifxetexorluatex
\ifxetex
  \xetexorluatextrue
\else
  \ifluatex
    \xetexorluatextrue
  \else
    \xetexorluatexfalse
  \fi
\fi
%
% \ifxetexorluatex%
%   \usepackage{fontspec}
%   \usepackage{libertine} % or use \setmainfont to choose any font on your system
%   \newfontfamily\quotefont[Ligatures=TeX]{Linux Libertine O} % selects Libertine as the quote font
% \else
%   \usepackage[utf8]{inputenc}
%   \usepackage[T1]{fontenc}
%   \usepackage{libertine} % or any other font package
%   \newcommand*\quotefont{\fontfamily{LinuxLibertineT-LF}} % selects Libertine as the quote font
% \fi

\newcommand*\quotesize{60} % if quote size changes, need a way to make shifts relative
% Make commands for the quotes
\newcommand*{\openquote}
   {\tikz[remember picture,overlay,xshift=-4ex,yshift=-2.5ex]
   \node (OQ) {\fontsize{\quotesize}{\quotesize}\selectfont``};\kern0pt}

\newcommand*{\closequote}[1]
  {\tikz[remember picture,overlay,xshift=4ex,yshift={#1}]
   \node (CQ) {\fontsize{\quotesize}{\quotesize}\selectfont''};}

% select a colour for the shading
\colorlet{shadecolor}{white}

\newcommand*\shadedauthorformat{\emph} % define format for the author argument

% Now a command to allow left, right and centre alignment of the author
\newcommand*\authoralign[1]{%
  \if#1l
    \def\authorfill{}\def\quotefill{\hfill}
  \else
    \if#1r
      \def\authorfill{\hfill}\def\quotefill{}
    \else
      \if#1c
        \gdef\authorfill{\hfill}\def\quotefill{\hfill}
      \else\typeout{Invalid option}
      \fi
    \fi
  \fi}
% wrap everything in its own environment which takes one argument (author) and one optional argument
% specifying the alignment [l, r or c]
%
\newenvironment{shadequote}[2][l]%
{\authoralign{#1}
\ifblank{#2}
   {\def\shadequoteauthor{}\def\yshift{-2ex}\def\quotefill{\hfill}}
   {\def\shadequoteauthor{\par\authorfill\shadedauthorformat{#2}}\def\yshift{2ex}}
\begin{snugshade}\begin{quote}\openquote}
{\shadequoteauthor\quotefill\closequote{\yshift}\end{quote}\end{snugshade}}
